\documentclass[11pt, letterpaper]{article}

% Packages for formatting and functionality
\usepackage[utf8]{inputenc}
\usepackage[T1]{fontenc}
\usepackage{mathptmx}       % Times New Roman font for a professional look
\usepackage{geometry}       % Page margins
\usepackage{xcolor}         % Colors for headings/links
\usepackage{titlesec}       % Custom section titles
\usepackage{fancyhdr}       % Headers and footers
\usepackage{graphicx}       % For including images/plots
\usepackage{hyperref}       % Hyperlinks
\usepackage{booktabs}       % Professional tables
\usepackage{setspace}       % Line spacing
\usepackage{caption}        % Caption formatting
\usepackage{float}          % Figure placement

% Page Geometry
\geometry{left=1in, right=1in, top=1in, bottom=1in}

% Color Definitions (Corporate/Financial Style)
\definecolor{navyblue}{RGB}{0, 51, 102}
\definecolor{darkgray}{RGB}{64, 64, 64}

% Typography and Spacing
\onehalfspacing
\setlength{\parskip}{0.5em}
\setlength{\parindent}{0pt}

% Hyperlink Setup
\hypersetup{
    colorlinks=true,
    linkcolor=navyblue,
    filecolor=navyblue,      
    urlcolor=navyblue,
    citecolor=navyblue
}

% Header and Footer Setup
\pagestyle{fancy}
\fancyhf{}
\lhead{\textcolor{darkgray}{\small \textit{Carvana Auto Receivables Trust Analysis}}}
\rhead{\textcolor{darkgray}{\small \today}}
\cfoot{\thepage}
\renewcommand{\headrulewidth}{0.5pt}
\renewcommand{\footrulewidth}{0.0pt}

% Section Heading Format
\titleformat{\section}
  {\Large\bfseries\color{navyblue}}
  {\thesection}{1em}{}
\titleformat{\subsection}
  {\large\bfseries\color{navyblue}}
  {\thesubsection}{1em}{}

% Title Page Information
\title{
    \vspace{1in}
    \Huge \textbf{\textcolor{navyblue}{Analysis of Delinquency Trends}} \\
    \vspace{0.2in}
    \Large \textit{\textcolor{darkgray}{A Deep Dive into Carvana Auto Receivables Trust (CVNA) Data}}
    \vspace{1in}
}
\author{
    \textbf{Tyler Lukasiewicz} \\
    \textit{Financial Analyst} \\
    \textit{Institution Name}
}
\date{\today}

\begin{document}

% Title Page
\maketitle
\thispagestyle{empty}
\newpage

% Abstract / Executive Summary
\begin{abstract}
    \noindent \textbf{Executive Summary:} This whitepaper presents a comprehensive analysis of the collateral characteristics and risk factors within Carvana Auto Receivables Trust (CVNA) securitizations from vintages 2021 through 2025. By leveraging loan-level data, we examine trends in Loan-to-Value (LTV) ratios, Payment-to-Income (PTI) ratios, and interest rates. Furthermore, we perform regression analyses to quantify the relationship between obligor credit scores and these risk metrics. 
    
    Key findings indicate a strong adherence to risk-based pricing, evidenced by high correlations between credit scores and interest rates. However, we observe a structural shift in collateral leverage, with recent vintages (2024-2025) exhibiting significantly higher LTV ratios for comparable credit profiles than their 2021 predecessors. Additionally, we track "risk layering" trends among prime borrowers to identify potential pockets of latent risk.
\end{abstract}

\tableofcontents
\newpage

\section{Introduction}
The securitization of auto loans is a critical component of the consumer credit market. This report focuses on the performance and composition of Carvana Auto Receivables Trusts (CVNA). Specifically, we analyze the evolution of underwriting standards over time and the correlation between borrower creditworthiness (FICO scores) and key loan attributes such as Loan-to-Value (LTV), Payment-to-Income (PTI), and Annual Percentage Rate (APR).

\section{Data and Methodology}
The analysis utilizes loan-level data stored in the \texttt{carvana\_assets} database, covering multiple trust vintages from 2021 to 2025. 

Data cleaning and preprocessing steps included:
\begin{itemize}
    \item \textbf{Exclusion of Inactive Loans:} Loans with a zero reporting period balance (paid-off or charged-off) were excluded to focus on the active portfolio.
    \item \textbf{Metric Calculation:} 
    \begin{itemize}
        \item \textit{Raw LTV}: Calculated as the Current Loan Balance divided by the Original Vehicle Value.
        \item \textit{PTI}: Payment-to-Income ratios were filtered to exclude invalid (zero) entries.
    \end{itemize}
    \item \textbf{Regression Analysis:} We employed linear regression models ($y = \beta_0 + \beta_1 x + \epsilon$) to determine the sensitivity of risk metrics ($y$) to the Obligor Credit Score ($x$).
\end{itemize}

\section{Results}

\subsection{Portfolio Trends}
We observed distinct trends in the composition of the trusts over time.

\begin{figure}[H]
    \centering
    \includegraphics[width=1.0\textwidth]{plots/trends/trend_ltv.png}
    \caption{Average LTV Trend by Trust Vintage}
    \label{fig:trend_ltv}
\end{figure}

\begin{figure}[H]
    \centering
    \includegraphics[width=1.0\textwidth]{plots/trends/trend_rate.png}
    \caption{Average Interest Rate Trend by Trust Vintage}
    \label{fig:trend_rate}
\end{figure}

\subsection{Regression Analysis}
To assess the efficacy of underwriting, we analyzed the correlation between credit scores and loan attributes.

\subsubsection{Credit Score vs. Payment-to-Income (PTI)}
The correlation between credit score and PTI remains consistently weak across all vintages, typically ranging between -0.13 and -0.25. The slopes are negligible (~-0.0001), indicating that a 100-point increase in FICO score results in only a ~1\% decrease in PTI. This suggests that high-PTI loans are distributed across the credit spectrum and are not strictly a function of credit score.

% Table 1: Credit Score vs PTI
\begin{table}[ht]
\centering
\caption{Regression Analysis: Credit Score vs. Payment-to-Income (PTI)}
\begin{tabular}{lcccc}
\hline
Trust & Slope & Intercept & $R^2$ & Correlation \\
\hline
2021-P1 & -0.000102 & 0.1515 & 0.0311 & -0.1764 \\
2021-P2 & -0.000128 & 0.1759 & 0.0477 & -0.2184 \\
2021-P3 & -0.000113 & 0.1660 & 0.0347 & -0.1864 \\
2021-P4 & -0.000114 & 0.1667 & 0.0358 & -0.1893 \\
2022-P1 & -0.000106 & 0.1592 & 0.0361 & -0.1899 \\
2022-P2 & -0.000113 & 0.1664 & 0.0345 & -0.1859 \\
2022-P3 & -0.000083 & 0.1429 & 0.0186 & -0.1365 \\
2024-P2 & -0.000104 & 0.1504 & 0.0330 & -0.1818 \\
2024-P3 & -0.000111 & 0.1552 & 0.0370 & -0.1924 \\
2024-P4 & -0.000097 & 0.1452 & 0.0285 & -0.1687 \\
2025-P1 & -0.000124 & 0.1612 & 0.0584 & -0.2417 \\
2025-P2 & -0.000128 & 0.1646 & 0.0603 & -0.2456 \\
2025-P3 & -0.000128 & 0.1639 & 0.0614 & -0.2477 \\
2025-P4 & -0.000121 & 0.1575 & 0.0592 & -0.2432 \\
\hline
\end{tabular}
\end{table}

\subsubsection{Credit Score vs. Loan-to-Value (LTV)}
A moderate negative correlation exists between credit scores and LTV ratios. However, the most significant finding is the shift in the \textbf{Intercept}. In 2021 vintages, the intercept was approximately 60-80. In 2024 and 2025 vintages, this intercept jumped to ranges of 110-135. This implies that for a fixed credit score, the starting LTV is significantly higher in recent vintages, suggesting a loosening of collateral requirements or an increase in vehicle valuations relative to loan amounts.

% Table 2: Credit Score vs LTV
\begin{table}[ht]
\centering
\caption{Regression Analysis: Credit Score vs. Loan-to-Value (LTV)}
\begin{tabular}{lcccc}
\hline
Trust & Slope & Intercept & $R^2$ & Correlation \\
\hline
2021-P1 & -0.054343 & 59.5945 & 0.1734 & -0.4164 \\
2021-P2 & -0.057040 & 65.8101 & 0.1771 & -0.4208 \\
2021-P3 & -0.060969 & 73.2892 & 0.1780 & -0.4219 \\
2021-P4 & -0.068606 & 82.6397 & 0.1936 & -0.4400 \\
2022-P1 & -0.065652 & 83.5368 & 0.1733 & -0.4162 \\
2022-P2 & -0.069353 & 89.7838 & 0.1471 & -0.3835 \\
2022-P3 & -0.065286 & 93.8294 & 0.1079 & -0.3285 \\
2024-P2 & -0.080788 & 129.7512 & 0.1001 & -0.3164 \\
2024-P3 & -0.080971 & 132.8273 & 0.0947 & -0.3077 \\
2024-P4 & -0.078225 & 135.0695 & 0.0818 & -0.2861 \\
2025-P1 & -0.037742 & 108.7702 & 0.0214 & -0.1462 \\
2025-P2 & -0.037953 & 113.1292 & 0.0238 & -0.1543 \\
2025-P3 & -0.038663 & 116.9933 & 0.0258 & -0.1606 \\
2025-P4 & -0.034865 & 117.1543 & 0.0217 & -0.1474 \\
\hline
\end{tabular}
\end{table}

\subsubsection{Credit Score vs. Interest Rate}
As expected, there is a strong negative correlation (typically stronger than -0.60) between credit scores and interest rates. This confirms that risk-based pricing is a dominant factor in loan structuring. The intercepts have increased from ~0.31 (31\%) in 2021 to ~0.40 (40\%) in 2025, reflecting the broader macroeconomic environment of rising interest rates.

% Table 3: Credit Score vs Interest Rate
\begin{table}[ht]
\centering
\caption{Regression Analysis: Credit Score vs. Interest Rate}
\begin{tabular}{lcccc}
\hline
Trust & Slope & Intercept & $R^2$ & Correlation \\
\hline
2021-P1 & -0.000363 & 0.3391 & 0.5387 & -0.7339 \\
2021-P2 & -0.000356 & 0.3309 & 0.5527 & -0.7434 \\
2021-P3 & -0.000335 & 0.3106 & 0.5189 & -0.7203 \\
2021-P4 & -0.000332 & 0.3142 & 0.5420 & -0.7362 \\
2022-P1 & -0.000361 & 0.3309 & 0.6073 & -0.7793 \\
2022-P2 & -0.000343 & 0.3331 & 0.5324 & -0.7296 \\
2022-P3 & -0.000315 & 0.3257 & 0.4801 & -0.6929 \\
2024-P2 & -0.000309 & 0.3571 & 0.3302 & -0.5746 \\
2024-P3 & -0.000309 & 0.3571 & 0.3346 & -0.5784 \\
2024-P4 & -0.000275 & 0.3272 & 0.2480 & -0.4980 \\
2025-P1 & -0.000377 & 0.4057 & 0.4800 & -0.6928 \\
2025-P2 & -0.000355 & 0.3791 & 0.4090 & -0.6396 \\
2025-P3 & -0.000333 & 0.3566 & 0.4114 & -0.6414 \\
2025-P4 & -0.000363 & 0.3706 & 0.4640 & -0.6812 \\
\hline
\end{tabular}
\end{table}

\subsection{Risk Layering (Danger Zones)}
We monitored specific segments of "Prime" borrowers (FICO $>$ 740) who exhibit potentially risky characteristics, specifically high PTI or high Interest Rates.

\begin{figure}[H]
    \centering
    \includegraphics[width=1.0\textwidth]{plots/trends/trend_danger_prime_pti.png}
    \caption{Trend: Prime Borrowers with High PTI ($>$ 12\%)}
    \label{fig:danger_pti}
\end{figure}

\begin{figure}[H]
    \centering
    \includegraphics[width=1.0\textwidth]{plots/trends/trend_danger_prime_rate.png}
    \caption{Trend: Prime Borrowers with High Interest Rates ($>$ 12\%)}
    \label{fig:danger_rate}
\end{figure}

\section{Conclusion}
In conclusion, the analysis of Carvana ABS trusts reveals distinct shifts in underwriting characteristics from 2021 to 2025. 

While risk-based pricing remains robust (evidenced by the strong Credit-Rate correlation), the structural shift in LTV intercepts indicates that recent vintages are more leveraged relative to borrower credit scores than in the past. The weak correlation in PTI suggests that affordability metrics are less sensitive to credit scores, potentially masking risk if economic conditions deteriorate for higher-income borrowers. The "Danger Zone" analysis further highlights that even prime borrowers are increasingly taking on loans with higher relative payments and rates.

\end{document}