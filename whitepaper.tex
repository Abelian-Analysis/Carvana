\documentclass[11pt, letterpaper]{article}

% Packages for formatting and functionality
\usepackage[utf8]{inputenc}
\usepackage[T1]{fontenc}
\usepackage{mathptmx}       
\usepackage{geometry}       
\usepackage[table]{xcolor}  
\usepackage{titlesec}       
\usepackage{fancyhdr}       
\usepackage{graphicx}       
\usepackage{hyperref}       
\usepackage{booktabs}       
\usepackage{setspace}       
\usepackage{caption}        
\usepackage{float}          

% Page Geometry
\geometry{left=1in, right=1in, top=1in, bottom=1in}

% Color Definitions
\definecolor{navyblue}{RGB}{0, 51, 102}
\definecolor{darkgray}{RGB}{64, 64, 64}

% Typography and Spacing
\onehalfspacing
\setlength{\parskip}{0.5em}
\setlength{\parindent}{0pt}

% Hyperlink Setup
\hypersetup{
    colorlinks=true,
    linkcolor=navyblue,
    urlcolor=navyblue
}

% Header and Footer Setup
\pagestyle{fancy}
\fancyhf{}
\setlength{\headheight}{14pt}
\lhead{\textcolor{darkgray}{\small \textit{SEC Whistleblower Submission: 
Carvana P-Series Structural Analysis}}}
\rhead{\textcolor{darkgray}{\small \today}}
\cfoot{\thepage}
\renewcommand{\headrulewidth}{0.5pt}

% Section Heading Format
\titleformat{\section}{\Large\bfseries\color{navyblue}}{\thesection}{1em}{}
\titleformat{\subsection}{\large\bfseries\color{navyblue}}{\thesubsection}{1em}{}

% Title Page
\title{
    \vspace{1in}
    \Huge \textbf{\textcolor{navyblue}{The Erosion of Prime:}} \\
    \Large \textbf{\textcolor{navyblue}{Systemic Data Obfuscation and Structural 
    Risk in Carvana P-Series Auto ABS}} \\
    \vspace{0.2in}
    \Large \textit{\textcolor{darkgray}{Formal Complaint regarding Carvana Auto 
    Receivables Trust (CVNA) Regulatory Compliance and Risk Disclosure}}
    \vspace{1in}
}
\author{
    \textbf{Tyler Lukasiewicz} \\
    \textit{Independent Quantitative Researcher} \\
    tlukasiewicz7993@gmail.com \\
    1 (561) 512-0837
}
\date{\today}

\begin{document}

\maketitle
\thispagestyle{empty}
\newpage

\section{Executive Summary}
This submission provides evidence that the issuer of the Carvana P-Series 
Asset-Backed Securities (ABS) is utilizing a "Prime" designation to market a 
portfolio that exhibits risk characteristics typically associated with subprime 
lending. Through 
quantitative analysis of Regulation AB II loan-level data, we identify a 
persistent pattern of \textbf{Data Integrity Failures}, \textbf{Collateral 
Deterioration}, and \textbf{Trigger Avoidance}. 

Central to this complaint is the revelation that the issuer maintains a 100\% 
"No-Doc" (Verification Code 3) profile across the pool, while simultaneously 
utilizing 73-month term extensions to mask a burgeoning collateral crisis where 
36.1\% of the loans are now "underwater." This confluence of unverified income 
and negative equity creates a systemic "Tail Risk" that remains inadequately 
disclosed to the investment community.


\section{Systemic Failure of Income Verification (Code 3)}
The foundational metric for any Prime ABS is the Payment-to-Income (PTI) ratio. 
In the P-Series, the issuer reports an attractive average PTI of \textbf{7.2\%}. 
However, the reliability of this figure is limited by the underlying data 
quality.

\begin{itemize}
    \item \textbf{Stated Income vs. Verified Reality:} 100\% of the loans in the 
    pool utilize \textbf{Income Verification Code 3 (Stated, Not Verified)}. 
    \item \textbf{Understatement of Default Probability:} By failing to verify 
    income, the issuer effectively bypasses the primary guardrail against loans 
    lacking income verification. If actual borrower income is lower than stated, 
    the true PTI is 
    significantly higher, rendering the "Prime" affordability narrative 
    materially misleading. This failure of input verification compromises the 
    integrity of downstream risk models, resulting in the pricing anomalies 
    observed in the P-Series.
\end{itemize}


\section{The Decomposition of Risk-Based Pricing}
A fundamental tenet of Prime ABS is the strong correlation between a borrower’s credit score (FICO) and the interest rate assigned to the loan. Linear regression analysis of the P-Series data demonstrates that this relationship has fundamentally collapsed, signaling that the ``Prime'' label is being used as a veneer for a more aggressive, non-traditional pricing model.

\subsection{Predictive Failure and Model Decoupling}
The $R^2$ coefficient, which measures how well the FICO score predicts the interest rate, has shown a catastrophic decline. In early vintages, FICO scores explained over 60\% of the pricing variance; by 2024, this plummeted to 24.8\%, indicating that ``hidden variables'' began to drive the issuer's risk assessment.

% INSERT TABLE 3 HERE
\begin{table}[ht]
\centering
\caption{Regression Analysis: FICO Score vs. Interest Rate Decoupling}
\begin{tabular}{lcccc}
\hline
\textbf{Trust} & \textbf{Slope} & \textbf{Intercept} & \textbf{$R^2$} & \textbf{Correlation} \\
\hline
2021-P1 & -0.000363 & 0.3391 & 0.5387 & -0.7339 \\
2022-P1 & -0.000361 & 0.3309 & 0.6073 & -0.7793 \\
\rowcolor{red!15} 2024-P2 & -0.000309 & 0.3571 & 0.3302 & -0.5746 \\
\rowcolor{red!25} 2024-P4 & -0.000275 & 0.3272 & \textbf{0.2480} & -0.4980 \\
2025-P4 & -0.000363 & 0.3706 & 0.4640 & -0.6812 \\
\hline
\end{tabular}
\end{table}

\subsection{Strategic "Yield Harvesting" and Payment Shock}
The breakdown in $R^2$ correlates precisely with a period of aggressive \textbf{Yield Harvesting}. As the Federal Reserve raised interest rates, many borrowers experienced ``Payment Shock''---a sudden contraction in disposable income as debt service costs rose across their entire financial profile.

Because FICO scores are trailing indicators, these borrowers maintained ``Prime'' scores (740+) even as their actual debt-capacity deteriorated. The issuer appears to have exploited this lag by originating loans at subprime rates for borrowers with prime scores.

\begin{figure}[H]
    \centering
    \includegraphics[width=0.9\textwidth]{plots/trends/trend_danger_prime_rate.png} % INSERT SPIKE GRAPH HERE
    \caption{Trend: Percentage of Prime Borrowers (FICO > 740) assigned High Interest Rates (> 12\%)}
    \label{fig:yield_harvest}
\end{figure}

\textbf{Analysis of the 8.47\% Spike:}
\begin{itemize}
    \item \textbf{Exploitation of Credit Lag:} Prior to 2024, high-rate Prime originations were negligible (<0.5\%). The spike to 8.47\% in 2024-P2 suggests a tactical pivot to capture high-yield assets from credit-fragile borrowers.
    \item \textbf{Incongruent Risk Pricing:} Charging a 12\% rate to a 740+ FICO borrower is an admission that the lender views the borrower's probability of default as subprime, despite the ``Prime'' designation of the security.
    \item \textbf{Mean Reversion:} The subsequent drop to 1.86\% in 2025-P4 suggests the issuer curtailed this practice only once the portfolio's aggregate delinquency triggers were threatened.
\end{itemize}

\section{The Failure of Risk-Standardization (Credit vs. Capacity)}
In a standard Prime lending environment, credit quality and leverage are inherently linked; higher FICO scores traditionally correlate with lower loan-to-value (LTV) ratios and lower payment-to-income (PTI) burdens. However, regression analysis of the 2025 P-Series trusts reveals a \textbf{near-total decoupling} of these metrics, suggesting that the ``Prime'' credit score is being used to mask subprime levels of leverage.

\subsection{The Decoupling of Credit and Affordability (PTI)}
As shown in Table 4, the FICO score has essentially no relationship with a borrower's reported debt-service capacity. With an $R^2$ of only \textbf{0.0592} in the 2025-P4 trust, less than 6\% of the variance in affordability is explained by the borrower’s credit score.

\begin{table}[ht]
\centering
\caption{Regression Analysis: Credit Score vs. Payment-to-Income (PTI)}
\begin{tabular}{lcccc}
\hline
\textbf{Trust} & \textbf{Slope} & \textbf{Intercept} & \textbf{$R^2$} & \textbf{Correlation} \\
\hline
2025-P3 & -0.000128 & 0.1639 & 0.0614 & -0.2477 \\
2025-P4 & -0.000121 & 0.1575 & 0.0592 & -0.2432 \\
\hline
\end{tabular}
\end{table}

\textbf{Forensic Implication:} The near-zero correlation suggests that ``Super-Prime'' borrowers are being qualified for the same high-PTI payment burdens as lower-tier borrowers. Given the \textbf{100\% reliance on Verification Code 3 (Stated Income)}, this lack of correlation suggests the income variable is being manipulated to force approvals for high-leverage loans that would otherwise fail a traditional Prime manual underwrite.

\subsection{The Decoupling of Credit and Collateral Security (LTV)}
The breakdown in the relationship between FICO and LTV is even more severe. Table 5 demonstrates that the borrower's credit score explains a negligible \textbf{2.17\%} of the LTV variance in the most recent vintage.

\begin{table}[ht]
\centering
\caption{Regression Analysis: Credit Score vs. Loan-to-Value (LTV)}
\begin{tabular}{lcccc}
\hline
\textbf{Trust} & \textbf{Slope} & \textbf{Intercept} & \textbf{$R^2$} & \textbf{Correlation} \\
\hline
2025-P3 & -0.038663 & 116.9933 & 0.0258 & -0.1606 \\
2025-P4 & -0.034865 & 117.1543 & 0.0217 & -0.1474 \\
\hline
\end{tabular}
\end{table}

\textbf{Forensic Implication:} The \textbf{Intercept of 117.15} for 2025-P4 indicates that even at the theoretical zero-point of the model, loans are originated deep in negative equity. The lack of correlation proves that \textbf{36.1\% underwater rates} are applied indiscriminately across the credit spectrum. 



\subsection{Conclusion: Targeting the ``Thin-File'' Profile}
The combination of these data points reveals a deliberate ``Cherry-Picking'' strategy. The issuer is targeting \textbf{``Score-Rich, Depth-Poor''} borrowers who possess a Prime score but lack the financial reserves to avoid \textbf{73-month terms} and \textbf{high-LTV} structures. This allows the issuer to harvest high yields from what are functionally subprime-level debt burdens while maintaining a high-FICO branding for the Trust.


\section{The "Underwater" Collateral Crisis}
The issuer has utilized extended loan terms as a mathematical lever to manage
monthly affordability at the direct expense of the underlying collateral
security. By slowing the rate of principal amortization while vehicle
depreciation remains constant, the issuer has structurally engineered a
portfolio with high loss severity.




\subsection{The Mechanism: Amortization vs. Depreciation}
The issuer’s primary objective in the 2024--2025 period was to maintain an
Average Payment-to-Income (PTI) ratio of approximately 7.2\% despite a rising
interest rate environment. To achieve this, the loan term ($n$) was increased to
a record 73 months.

\begin{figure}[H]
    \centering
    \includegraphics[width=1.0\textwidth]{plots/trends/trend_term.png}
    \caption{Trend: Average Original Loan Term (Months)}
\end{figure}

\textbf{Structural Default Trap:} With an average term of \textbf{73 
months}, the principal pay-down is too slow to outpace vehicle depreciation. 
This shift results in a collateral deficiency where the outstanding loan 
balance exceeds the asset value, significantly increasing the Loss Given 
Default (LGD).

The monthly payment ($PMT$) is governed by the standard annuity formula:
\[ PMT = P \cdot \frac{r(1+r)^n}{(1+r)^n - 1} \]

Where:
\begin{itemize}
    \item $P$ = Principal (Original Loan Amount)
    \item $r$ = Monthly Interest Rate
    \item $n$ = Loan Term in Months
\end{itemize}

By increasing $n$ from 71 to 73, the issuer successfully reduced the $PMT$
required for a given principal, thereby maintaining the "Prime" appearance of
affordability.

\subsection{The Resultant Equity Gap}
The risk to the Trust is defined by the Loan-to-Value ($LTV$) ratio at any time $t$:
\[ LTV_t = \frac{Balance_t}{Value_t} \]

While $Balance_t$ is reduced more slowly due to the 73-month term extension, the
vehicle value ($Value_t$) continues to follow a standard exponential
depreciation curve:
\[ Value_t = Value_0 \cdot e^{-kt} \]

Where $k$ represents the depreciation constant. Because the amortization curve
has been flattened by the term extension, $Balance_t$ remains higher than
$Value_t$ for a significantly longer duration. This mathematical "crossover" has
resulted in \textbf{36.1\% of the pool becoming underwater (LTV > 100\%)} as of
late 2025.

\begin{figure}[H]
    \centering
    \includegraphics[width=1.0\textwidth]{plots/trends/trend_ltv_underwater.png}
    \caption{Trend: Percentage of Underwater Loans (LTV > 100\%)}
\end{figure}

\textbf{The Duration-Severity Paradox:}
\begin{itemize}
    \item \textbf{Delayed Equity Recovery:} In a pool of over 35,000
    receivables, a 2-month average term increase represents a systemic deferral
    of principal repayment, pushing the "equity inflection point" dangerously
    far into the loan lifecycle.
    \item \textbf{Heightened Loss Given Default (LGD):} For the 36.1\% of loans
    currently underwater, any default event carries a significantly higher loss
    severity than historical Prime benchmarks.
    \item \textbf{Adverse Incentive:} High LTV ratios incentivize "strategic
    defaults," where borrowers with negative equity choose to abandon the asset
    during periods of economic stress or required maintenance.
\end{itemize}

\begin{figure}[H]
    \centering
    \includegraphics[width=1.0\textwidth]{plots/ltv_distros/Carvana_Auto_Receivables_Trust_2025_P4_ltv.png}
    \caption{LTV Distribution for 2025-P4: Structural Negative Equity}
\end{figure}

\textbf{Current Stricture Analysis (2025-P4):}
The distribution of Loan-to-Value (LTV) ratios for the most recent vintage, 
2025-P4, illustrates a structural shift towards negative equity at origination. 
A significant portion of the pool is "underwater" (LTV > 100\%) from day one, 
indicating that the loan amounts exceed the value of the collateral. This 
structure relies heavily on borrower repayment continuity rather than asset 
recovery value.

\textbf{Conclusion:} The issuer has effectively traded \textbf{Collateral
Integrity} for \textbf{Payment Appearance}. This structural choice has replaced
borrower-based risk with asset-based risk, a shift that is not reflected in the
"P-Series" Prime designation.
With the collateral effectively hollowed out, the portfolio faces inevitable 
performance pressure. The following section demonstrates how the issuer actively 
masks this deterioration to avoid contractual consequences.

\section{Strategic Suppression of Contractual Triggers}

\begin{figure}[H]
    \centering
    \includegraphics[width=0.85\textwidth]{plots/Carvana_Auto_Receivables_Trust_2021_P2_delinquency.png}
    \caption{Delinquency vs. Extensions: 60+ Day Delinquency Breach Analysis}
\end{figure}

Significant evidence of potential risk masking is found in the correlation 
between delinquency triggers and aggressive loan extensions.

\textbf{The Deferral Mechanism:} The issuer utilizes aggressive loan extensions 
to artificially maintain the 60+ Day Delinquency rate below the 4.5\% threshold. 

\textbf{Forensic Analysis:} As of late 2025, the reported delinquency rate was 
\textbf{4.3\%}, yet the \textbf{Adjusted Delinquency Rate (inclusive of 
deferrals)} stood at \textbf{5.2\%}. This constitutes a \textbf{De Facto Breach} 
of the contractual Delinquency Trigger.

\textbf{The "Clustering" Evidence:} High-resolution analysis of the 2021-P2 
trust reveals a high-frequency clustering of extensions (grey bars) occurring 
immediately prior to the delinquency line (green) approaching the 4.5\% red 
trigger line. 

This pattern indicates that extensions are being deployed 
as a tactical response to avoid "Early Amortization" events. By masking the true 
loss severity, the issuer is withholding the mandatory "protective diversions" 
of cash flow required by the Trust Indenture, thereby prioritizing issuer 
liquidity over bondholder security.
The behaviors documented in the 2021-P2 trust are not isolated anomalies.
Cross-trust analysis of the P-Series\footnote{Additional graphs are amended to the filed complaint} reveals a consistent operational strategy:
when organic delinquency levels approach contractual triggers, extension volume
increases proportionately to maintain technical compliance while masking
economic failure.



\section{Conclusion: Evidence of Calculated Risk Concealment}


The quantitative data suggests that the issuer did not merely react to a
high-interest-rate environment, but actively exploited it. The 8.47\% spike in
high-rate Prime originations in 2024-P2 confirms a strategy of "Yield
Harvesting" from borrowers whose FICO scores masked underlying "Payment Shock."

\begin{itemize}
    \item \textbf{Knowledge of Deterioration:} The subsequent "turning off the
    tap" in 2025—where high-rate Prime originations plummeted back to
    1.86\%—coincides exactly with the period where \textbf{Delinquencies +
    Extensions} began threatening the 4.5\% contractual trigger.
    \item \textbf{Pattern of Recklessness:} The issuer utilized unverified
    "Stated" income (Code 3) and 73-month term extensions to maintain loan
    volume while 36.1\% of the pool fell underwater.
    \item \textbf{The Final Concealment:} When these structurally flawed loans
    began to fail, the issuer deployed aggressive extensions to mask the 5.2\%
    true delinquency rate.
\end{itemize}

\textbf{Final Assertion:} The "P-Series" Prime designation is a material
misrepresentation. The issuer has demonstrated an ongoing pattern of
prioritizing yield and volume over the structural integrity of the collateral
and the accuracy of investor disclosures.

\section{Conclusion and SEC Action Requested}
The Carvana P-Series exhibits a structural risk profile that is no longer 
consistent with a "Prime" classification. The combination of \textbf{Zero-Doc 
Lending}, \textbf{36\% Negative Equity}, and \textbf{Active Trigger Avoidance} 
constitutes a failure of disclosure under Regulation AB II.

\textbf{We respectfully request:}
\begin{enumerate}
    \item A formal investigation into the validity of "Code 3" income entries.
    \item An audit of the "Extension" policy to determine if deferrals are being 
    used to circumvent contractual performance triggers.
    \item A mandatory restatement of risk-weighting to account for the 73-month 
    term extension's impact on collateral recovery values.
\end{enumerate}

\section{Apendix}
\subsection{Data Sources}
\begin{itemize}
\item All data used in this report was sourced from 10-D and ABS-EE filings
EDGAR. All statistics, figures, and tables were generated by analyzing data
submitted by Carvana to the SEC.

\item An LLM was used to parse delinquency and extension data from over 400 10-D
filings. The raw data created by the LLM was manually audited for accuracy and 
modified in a few places where the LLM misread the tables in the filings. Both 
the raw data and modified data files are included with this submission.

\item All scipts used to pull data from EDGAR, to process the data, and to generate
the figures and tables in this report are included in the submission.
\end{itemize}

\subsection{Data Hygiene}
The quantitative findings in this report are derived from a granular analysis of
Reg AB II loan-level data. To ensure the statistical relevance of risk metrics,
the following data hygiene protocol was implemented in R:

\begin{itemize}
    \item \textbf{Active Receivables Isolation:} The dataset was filtered to
    exclude liquidated and paid-off loans. Specifically, any record where the
    \texttt{reportingPeriodBeginningLoanBalanceAmount} was less than or equal to
    \$0.01 was removed.
    \item \textbf{Rationale for Filtering:} This exclusion is necessary to
    prevent "Survival Bias" from diluting point-in-time risk assessments.
    Including zero-balance loans would artificially lower the \textbf{Weighted
    Average LTV} and \textbf{Delinquency Percentages}, as these loans no longer
    contribute to the pool's exposure to loss.
    \item \textbf{Statistical Integrity:} By restricting the scope to active
    receivables, the resulting $R^2$ values and PTI trends reflect the
    \textbf{live economic reality} of the portfolio's current interest-bearing
    assets.
\end{itemize}

\end{document}