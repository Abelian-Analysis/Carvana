\documentclass[11pt, letterpaper]{article}

% Packages for formatting and functionality
\usepackage[utf8]{inputenc}
\usepackage[T1]{fontenc}
\usepackage{mathptmx}       
\usepackage{geometry}       
\usepackage[table]{xcolor}  
\usepackage{titlesec}       
\usepackage{fancyhdr}       
\usepackage{graphicx}       
\usepackage{hyperref}       
\usepackage{booktabs}       
\usepackage{setspace}       
\usepackage{caption}        
\usepackage{float}          

% Page Geometry
\geometry{left=1in, right=1in, top=1in, bottom=1in}

% Color Definitions
\definecolor{navyblue}{RGB}{0, 51, 102}
\definecolor{darkgray}{RGB}{64, 64, 64}

% Typography and Spacing
\onehalfspacing
\setlength{\parskip}{0.5em}
\setlength{\parindent}{0pt}

% Hyperlink Setup
\hypersetup{
    colorlinks=true,
    linkcolor=navyblue,
    urlcolor=navyblue
}

% Header and Footer Setup
\pagestyle{fancy}
\fancyhf{}
\setlength{\headheight}{14pt}
\lhead{\textcolor{darkgray}{\small \textit{SEC Whistleblower Submission: 
Carvana P-Series Structural Analysis}}}
\rhead{\textcolor{darkgray}{\small \today}}
\cfoot{\thepage}
\renewcommand{\headrulewidth}{0.5pt}

% Section Heading Format
\titleformat{\section}{\Large\bfseries\color{navyblue}}{\thesection}{1em}{}
\titleformat{\subsection}{\large\bfseries\color{navyblue}}{\thesubsection}{1em}{}

% Title Page
\title{
    \vspace{1in}
    \Huge \textbf{\textcolor{navyblue}{The Erosion of Prime:}} \\
    \Large \textbf{\textcolor{navyblue}{Systemic Data Obfuscation and Structural 
    Risk in Carvana P-Series Auto ABS}} \\
    \vspace{0.2in}
    \Large \textit{\textcolor{darkgray}{Formal Complaint regarding Carvana Auto 
    Receivables Trust (CVNA) Regulatory Compliance and Risk Disclosure}}
    \vspace{1in}
}
\author{
    \textbf{Tyler Lukasiewicz} \\
    \textit{Independent Quantitative Researcher} \\
    tlukasiewicz7993@gmail.com \\
    1 (561) 512-0837
}
\date{\today}

\begin{document}

\maketitle
\thispagestyle{empty}
\newpage

\section{Executive Summary}
This submission provides evidence that the issuer of the Carvana P-Series 
Asset-Backed Securities (ABS) is utilizing a "Prime" designation to market a 
portfolio that exhibits risk characteristics typically associated with subprime 
lending. Through 
quantitative analysis of Regulation AB II loan-level data, we identify a 
persistent pattern of \textbf{Data Integrity Failures}, \textbf{Collateral 
Deterioration}, and \textbf{Trigger Avoidance}. 

Central to this complaint is the revelation that the issuer maintains a 100\% 
"No-Doc" (Verification Code 3) profile across the pool, while simultaneously 
utilizing 73-month term extensions to mask a burgeoning collateral crisis where 
36.1\% of the loans are now "underwater." This confluence of unverified income 
and negative equity creates a systemic "Tail Risk" that remains inadequately 
disclosed to the investment community.


\section{Systemic Failure of Income Verification (Code 3)}
The foundational metric for any Prime ABS is the Payment-to-Income (PTI) ratio. 
In the P-Series, the issuer reports an attractive average PTI of \textbf{7.2\%}. 
However, the reliability of this figure is limited by the underlying data 
quality.

\begin{itemize}
    \item \textbf{Stated Income vs. Verified Reality:} 100\% of the loans in the 
    pool utilize \textbf{Income Verification Code 3 (Stated, Not Verified)}. 
    \item \textbf{Understatement of Default Probability:} By failing to verify 
    income, the issuer effectively bypasses the primary guardrail against loans 
    lacking income verification. If actual borrower income is lower than stated, 
    the true PTI is 
    significantly higher, rendering the "Prime" affordability narrative 
    materially misleading.
\end{itemize}

\section{Predictive Failure and Model Decoupling}
Regression analysis of the Credit Score vs. Interest Rate relationship reveals 
a fundamental breakdown in the issuer's risk-based pricing model.

\begin{table}[H]
\centering
\caption{Regression Analysis: Evidence of Pricing Decoupling}
\begin{tabular}{lcccc}
\toprule
\textbf{Trust} & \textbf{Slope} & \textbf{Intercept} & \textbf{$R^2$} & \textbf{Correlation} \\
\midrule
2021-P1 & -0.000363 & 0.3391 & 0.5387 & -0.7339 \\
2022-P1 & -0.000361 & 0.3309 & 0.6073 & -0.7793 \\
\rowcolor{red!15} 2024-P4 & -0.000275 & 0.3272 & \textbf{0.2480} & -0.4980 \\
2025-P4 & -0.000363 & 0.3706 & 0.4640 & -0.6812 \\
\bottomrule
\end{tabular}
\end{table}

\textbf{Statistical Analysis:} A Prime pool should maintain high predictability 
($R^2 > 0.50$). The crash to \textbf{0.2480} in 2024-P4 indicates that "Hidden 
Variables"—unrelated to borrower creditworthiness—are driving the pricing. 
While the issuer may cite Fed rate hikes as the cause, the data suggests the 
Fed hikes coincided with \textbf{Risk Layering} (extending terms to 73 
months to maintain loan volume).

\section{The Collateral Crisis (LTV and Negative Equity)}
The issuer has leveraged extended loan terms to manage monthly affordability at 
the direct expense of the underlying collateral security.

\begin{itemize}
    \item \textbf{The Underwater Threshold:} The percentage of loans with a 
    Loan-to-Value (LTV) ratio > 100\% exploded from \textbf{0.9\% to 36.1\%} in 
    18 months.
\begin{figure}[H]
    \centering
    \includegraphics[width=1.0\textwidth]{plots/trends/trend_ltv_underwater.png}
    \caption{Trend: Percentage of Underwater Loans (LTV > 100\%)}
\end{figure}
    \item \textbf{Structural Default Trap:} With an average term of \textbf{73 
    months}, the principal pay-down is too slow to outpace vehicle depreciation. 
    This shift results in a collateral deficiency where the outstanding loan 
    balance exceeds the asset value, significantly increasing the Loss Given 
    Default (LGD).
\begin{figure}[H]
    \centering
    \includegraphics[width=1.0\textwidth]{plots/trends/trend_term.png}
    \caption{Trend: Average Original Loan Term (Months)}
\end{figure}
\end{itemize}



\textbf{Current Structure Analysis (2025-P4):}
The distribution of Loan-to-Value (LTV) ratios for the most recent vintage, 
2025-P4, illustrates a structural shift towards negative equity at origination. 
A significant portion of the pool is "underwater" (LTV > 100\%) from day one, 
indicating that the loan amounts exceed the value of the collateral. This 
structure relies heavily on borrower repayment continuity rather than asset 
recovery value.

\begin{figure}[H]
    \centering
    \includegraphics[width=1.0\textwidth]{plots/ltv_distros/Carvana_Auto_Receivables_Trust_2025_P4_ltv.png}
    \caption{LTV Distribution for 2025-P4: Structural Negative Equity}
\end{figure}


\section{Strategic Suppression of Contractual Triggers}

\begin{figure}[H]
    \centering
    \includegraphics[width=0.85\textwidth]{plots/Carvana_Auto_Receivables_Trust_2021_P2_delinquency.png}
    \caption{Delinquency vs. Extensions: 60+ Day Delinquency Breach Analysis}
\end{figure}

Significant evidence of potential risk masking is found in the correlation 
between delinquency triggers and aggressive loan extensions.

\textbf{The Deferral Mechanism:} The issuer utilizes aggressive loan extensions 
to artificially maintain the 60+ Day Delinquency rate below the 4.5\% threshold. 

\textbf{Forensic Analysis:} As of late 2025, the reported delinquency rate was 
\textbf{4.3\%}, yet the \textbf{Adjusted Delinquency Rate (inclusive of 
deferrals)} stood at \textbf{5.2\%}. This constitutes a \textbf{De Facto Breach} 
of the contractual Delinquency Trigger.

\textbf{The "Clustering" Evidence:} High-resolution analysis of the 2021-P2 
trust reveals a high-frequency clustering of extensions (grey bars) occurring 
immediately prior to the delinquency line (green) approaching the 4.5\% red 
trigger line. 

This pattern indicates that extensions are being deployed 
as a tactical response to avoid "Early Amortization" events. By masking the true 
loss severity, the issuer is withholding the mandatory "protective diversions" 
of cash flow required by the Trust Indenture, thereby prioritizing issuer 
liquidity over bondholder security.
The behaviors documented in the 2021-P2 trust are not isolated anomalies.
Cross-trust analysis of the P-Series\footnote{Additional graphs are amended to the filed complaint} reveals a consistent operational strategy:
when organic delinquency levels approach contractual triggers, extension volume
increases proportionately to maintain technical compliance while masking
economic failure.

\section{Conclusion and SEC Action Requested}
The Carvana P-Series exhibits a structural risk profile that is no longer 
consistent with a "Prime" classification. The combination of \textbf{Zero-Doc 
Lending}, \textbf{36\% Negative Equity}, and \textbf{Active Trigger Avoidance} 
constitutes a failure of disclosure under Regulation AB II.

\textbf{We respectfully request:}
\begin{enumerate}
    \item A formal investigation into the validity of "Code 3" income entries.
    \item An audit of the "Extension" policy to determine if deferrals are being 
    used to circumvent contractual performance triggers.
    \item A mandatory restatement of risk-weighting to account for the 73-month 
    term extension's impact on collateral recovery values.
\end{enumerate}

\section{Apendix}
\subsection{Data Sources}
\begin{itemize}
\item All data used in this report was sourced from 10-D and ABS-EE filings
EDGAR. All statistics, figures, and tables were generated by analyzing data
submitted by Carvana to the SEC.

\item An LLM was used to parse delinquency and extension data from over 400 10-D
filings. The raw data created by the LLM was manually audited for accuracy and 
modified in a few places where the LLM misread the tables in the filings. Both 
the raw data and modified data files are included with this submission.
the Github repository.

\item All scipts used to pull data from EDGAR, to process the data, and to generate
the figures and tables in this report are included in the submission.
\end{itemize}

\subsection{Data Hygiene}
The quantitative findings in this report are derived from a granular analysis of
Reg AB II loan-level data. To ensure the statistical relevance of risk metrics,
the following data hygiene protocol was implemented in R:

\begin{itemize}
    \item \textbf{Active Receivables Isolation:} The dataset was filtered to
    exclude liquidated and paid-off loans. Specifically, any record where the
    \texttt{reportingPeriodBeginningLoanBalanceAmount} was less than or equal to
    \$0.01 was removed.
    \item \textbf{Rationale for Filtering:} This exclusion is necessary to
    prevent "Survival Bias" from diluting point-in-time risk assessments.
    Including zero-balance loans would artificially lower the \textbf{Weighted
    Average LTV} and \textbf{Delinquency Percentages}, as these loans no longer
    contribute to the pool's exposure to loss.
    \item \textbf{Statistical Integrity:} By restricting the scope to active
    receivables, the resulting $R^2$ values and PTI trends reflect the
    \textbf{live economic reality} of the portfolio's current interest-bearing
    assets.
\end{itemize}

\end{document}