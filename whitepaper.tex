\documentclass[11pt, letterpaper]{article}

% Packages for formatting and functionality
\usepackage[utf8]{inputenc}
\usepackage[T1]{fontenc}
\usepackage{mathptmx}       
\usepackage{geometry}       
\usepackage[table]{xcolor}  
\usepackage{titlesec}       
\usepackage{fancyhdr}       
\usepackage{graphicx}       
\usepackage{hyperref}       
\usepackage{booktabs}       
\usepackage{setspace}       
\usepackage{caption}        
\usepackage{float}          

% Page Geometry
\geometry{left=1in, right=1in, top=1in, bottom=1in}

% Color Definitions
\definecolor{navyblue}{RGB}{0, 51, 102}
\definecolor{darkgray}{RGB}{64, 64, 64}

% Typography and Spacing
\onehalfspacing
\setlength{\parskip}{0.5em}
\setlength{\parindent}{0pt}

% Hyperlink Setup
\hypersetup{
    colorlinks=true,
    linkcolor=navyblue,
    urlcolor=navyblue
}

% Header and Footer Setup
\pagestyle{fancy}
\fancyhf{}
\setlength{\headheight}{14pt}
\lhead{\textcolor{darkgray}{\small \textit{SEC TCR Submission: 
Carvana P-Series}}}
\rhead{\textcolor{darkgray}{\small \today}}
\cfoot{\thepage}
\renewcommand{\headrulewidth}{0.5pt}

% Section Heading Format
\titleformat{\section}{\Large\bfseries\color{navyblue}}{\thesection}{1em}{}
\titleformat{\subsection}{\large\bfseries\color{navyblue}}{\thesubsection}{1em}{}

% Title Page
\title{
    \vspace{1in}
    \Huge \textbf{\textcolor{navyblue}{Subprime-Like Risk Characteristics in a Prime-Designated Auto Loan Securitization:}} \\
    \Large \textbf{\textcolor{navyblue}{Carvana P-Series}} \\
    \vspace{0.2in}
    \Large \textit{\textcolor{darkgray}{Formal Complaint regarding Carvana Auto 
    Receivables Trust (CVNA) Regulatory Compliance and Risk Disclosure}}
    \vspace{1in}
}
\author{
    \textbf{Tyler Lukasiewicz} \\
    \textit{Independent Quantitative Researcher} \\
}
\date{\today}

\begin{document}

\maketitle
\thispagestyle{empty}
\newpage

\section{Executive Summary}
This submission provides comprehensive quantitative evidence of systemic risk
factors within the Carvana P-Series Asset-Backed Securities (ABS). Our analysis
of Regulation AB II loan-level data reveals that the "P-Series" (Prime)
designation is statistically divergent from the underlying economic reality of
the collateral. 

We identify four critical pillars of structural instability that suggest the
portfolio exhibits subprime risk characteristics within a prime disclosure framework:

\begin{itemize}
    \item \textbf{Income Verification Patterns:} A 100\% utilization of "No-Doc"
    (Verification Code 3) income reporting. By relying exclusively on unverified
    "Stated Income," the issuer facilitates the reporting of lower
    Payment-to-Income (PTI) ratios, obscuring the true debt-service
    burden of the borrower pool.
    \item \textbf{Utilization of Credit Lag:} Quantitative
    evidence that the issuer is originating loans for high-score but low-liquidity
    individuals, realizing higher yields (APR > 12\%) from borrowers who carry
    a nominal Prime label.
    \item \textbf{Collateral Performance and the Equity Gap:} The strategic
    extension of loan terms to 73 months. This duration-extension mechanism has
    created a structural risk factor where 36.1\% of the pool is currently
    underwater ($LTV > 100\%$), as principal amortization fails to outpace
    vehicle depreciation.
    \item \textbf{Trigger Management Patterns:} Statistical clustering of loan
    extensions (deferrals) occurring immediately prior to 60-day delinquency
    breaches. This suggests an active management strategy to 
    maintain performance metrics below the 4.5\% contractual threshold, thereby
    bypassing mandatory investor protections.
\end{itemize}

This confluence of factors constitutes \textbf{Risk Arbitrage through Information Asymmetry}, presenting a material threat to the transparency and stability of the auto ABS market.

\section{Analysis of Stated Income Verification }

The foundational mechanism of the observed anomalies is the systemic reliance on stated
income verification, which serves to obscure the true debt-to-income burden of
the underlying borrowers. At the surface level, the issuer markets the P-Series
trusts by highlighting an attractive average Payment-to-Income (PTI) ratio of
\textbf{7.2\%}. However, the validity of this metric is immediately compromised
by the issuer's own reporting. According to the loan-level data, \textbf{100\%
of the loans} in these pools utilize \textbf{Income Verification Code 3}, which
denotes that income was merely ``Stated'' by the borrower and never
independently verified by the lender. This total reliance on unverified data
points transforms the PTI ratio from an objective risk metric into a subjective,
and potentially unreliable, assertion.

\section{The Decomposition of Risk-Based Pricing}
A fundamental tenet of Prime ABS is the predictive power of a borrower’s credit
score (FICO) in determining the interest rate (APR). In a stable prime
environment, FICO and APR should exhibit a strong inverse correlation. However,
linear regression analysis of the P-Series data demonstrates that this
relationship has significantly weakened, signaling that the ``Prime'' label is
being used to obscure a non-traditional, high-risk pricing model.
\subsection{Predictive Divergence and Credit Score Efficacy}
The $  R^2  $ coefficient, which measures the proportion of variance in interest
rates explained by FICO scores, has undergone a significant decline. This
suggests the emergence of credit-divergent borrowers—individuals who
maintain high FICO scores due to credit lag but lack the underlying cash flow or
collateral equity to support prime-level pricing.
\begin{itemize}
\item \textbf{Model Decoupling:} In 2022-P1, FICO scores explained
approximately 61\% of the pricing variance ($  R^2 = 0.6073  $). By the 2024-P4
vintage, this decreased to \textbf{24.8\%}.
\item \textbf{Latent Risk Variables:} This decoupling indicates that the
issuer's internal pricing engine has moved away from FICO as a primary risk
determinant, instead relying on undisclosed "latent variables" that more
closely resemble subprime underwriting.
\end{itemize}

% Table 3: Credit Score vs Interest Rate
\begin{table}[ht]
\centering
\caption{Regression Analysis: Credit Score vs. Interest Rate}
\begin{tabular}{lcccc}
\hline
Trust & Slope & Intercept & $R^2$ & Correlation \\
\hline
2022-P1 & -0.000361 & 0.3309 & 0.6073 & -0.7793 \\
2022-P2 & -0.000343 & 0.3331 & 0.5324 & -0.7296 \\
\rowcolor{red!7} 2022-P3 & -0.000315 & 0.3257 & 0.4801 & -0.6929 \\
\rowcolor{red!14} 2024-P2 & -0.000309 & 0.3571 & 0.3302 & -0.5746 \\
\rowcolor{red!21} 2024-P3 & -0.000309 & 0.3571 & 0.3346 & -0.5784 \\
\rowcolor{red!30} 2024-P4 & -0.000275 & 0.3272 & \textbf{0.2480} & -0.4980 \\
2025-P1 & -0.000377 & 0.4057 & 0.4800 & -0.6928 \\
2025-P2 & -0.000355 & 0.3791 & 0.4090 & -0.6396 \\
2025-P3 & -0.000333 & 0.3566 & 0.4114 & -0.6414 \\
2025-P4 & -0.000363 & 0.3706 & 0.4640 & -0.6812 \\
\hline
\end{tabular}
\end{table}


\subsection{Adverse Selection and Borrower Liquidity Profiles}
The sharp decline in the $R^2$ between FICO scores and APR from 2022-P1 (0.6073)
to 2024-P4 (0.2480) reflects a fundamental shift in Carvana's pricing engine
during the high-rate environment. Rather than relying primarily on credit scores
to determine risk-based pricing, the issuer increasingly assigned high APRs
(>12\%) to borrowers with prime FICO scores (>740). 

\begin{figure}[H]
\centering
\includegraphics[width=1.0\textwidth]{plots/trends/trend_danger_prime_rate.png}
\caption{Prime Borrowers with Subprime-Level APRs}
\end{figure}


Charging a subprime-level interest rate to a high-FICO borrower is effectively
an admission that the credit score is no longer a reliable predictor of
Probability of Default (PD) in these pools. By continuing to label these loans
as ``Prime,'' Carvana captured subprime-like yields while benefiting from the
lower capital requirements, higher ratings, and greater investor confidence
associated with prime ABS tranches.

This creates a clear potential for mispricing of risk: the ``Prime'' designation
masks the fragility of borrowers who maintain high credit scores (often due to
credit lag) but exhibit low liquidity, high debt-service burdens, or
insufficient collateral equity in a high-inflation, elevated-vehicle-price
environment.

From 2022 through much of 2024, this decoupling represented the primary
mechanism for maintaining attractive yields in a stressed macro backdrop.
However, by late 2024 and into 2025, market conditions began to shift: benchmark
interest rates stabilized and then declined, new- and used-vehicle inventory
normalized, and competitive pressure from traditional lenders intensified. These
changes reduced the need—and feasibility—of routinely assigning >12\% APRs to
prime-score borrowers, leading to a partial normalization of the FICO-APR
relationship in the 2025 vintages .


To sustain borrower affordability and issuer cash flow without reverting to the
earlier pricing mismatches, Carvana adapted by leaning more heavily on a
different lever: extending original loan terms. The following section examines
how this strategic pivot—while superficially improving pricing integrity—has
come at the direct expense of collateral quality, structurally increasing Loss
Given Default (LGD) through widespread negative equity.

\section{The "Underwater" Collateral Crisis}
The issuer has utilized extended loan terms as a mathematical lever to manage
monthly affordability at the direct expense of the underlying collateral
security. By slowing the rate of principal amortization while vehicle
depreciation remains constant, the issuer has structurally created a
portfolio with high loss severity.
\begin{figure}[H]
\centering
\includegraphics[width=1.0\textwidth]{plots/trends/trend_term.png}
\caption{Trend: Average Original Loan Term (Months)}
\end{figure}
Since the beginning of 2025, the average loan term has increased from 71 to 73
months. This sharp increase in weighted average loan terms represents a
systemic shift to manage borrower cash-flow constraints at the direct expense
of collateral amortization. The extension of loan duration
serves as the primary mechanism for preventing immediate liquidity-driven
defaults in a high-inflation environment.

\textbf{Structural Risk Factor:} With an average term of \textbf{73
months}, the principal pay-down is too slow to outpace vehicle depreciation.
This shift results in a collateral deficiency where the outstanding loan
balance exceeds the asset value, significantly increasing the Loss Given
Default (LGD).
The risk to the Trust is defined by the Loan-to-Value ($  LTV  $) ratio at any time $  t  $:
$   LTV_t = \frac{Balance_t}{Value_t}   $
While $  Balance_t  $ is reduced more slowly due to the 73-month term extension, the
vehicle value ($  Value_t  $) continues to follow a standard exponential
depreciation curve:
$   Value_t = Value_0 \cdot e^{-kt}   $
Where $  k  $ represents the depreciation constant. Because the amortization curve
has been flattened by the term extension, $  Balance_t  $ remains higher than
$  Value_t  $ for a significantly longer duration. This mathematical "crossover" has
resulted in \textbf{36.1\% of the pool becoming underwater (LTV > 100\%)} as of
late 2025.
\begin{figure}[H]
\centering
\includegraphics[width=1.0\textwidth]{plots/trends/trend_ltv_underwater.png}
\caption{Trend: Percentage of Underwater Loans (LTV > 100\%)}
\end{figure}
\textbf{The Duration-Severity Paradox:}
\begin{itemize}
\item \textbf{Delayed Equity Recovery:} In a pool of over 35,000
receivables, a 2-month average term increase represents a systemic deferral
of principal repayment, pushing the "equity inflection point" dangerously
far into the loan lifecycle.
\item \textbf{Heightened Loss Given Default (LGD):} For the 36.1\% of loans
currently underwater, any default event carries a significantly higher loss
severity than historical Prime benchmarks.
\item \textbf{Adverse Incentive:} High LTV ratios incentivize "strategic
defaults," where borrowers with negative equity choose to abandon the asset
during periods of economic stress or required maintenance.
\end{itemize}
\begin{figure}[H]
\centering
\includegraphics[width=1.0\textwidth]{plots/ltv_distros/Carvana_Auto_Receivables_Trust_2025_P4_ltv.png}
\caption{LTV Distribution for 2025-P4: Structural Negative Equity}
\end{figure}
The distribution of Loan-to-Value (LTV) ratios for the most recent vintage,
2025-P4, illustrates a structural shift towards negative equity at origination.
A significant portion of the pool is "underwater" (LTV > 100\%) from day one,
indicating that the loan amounts exceed the value of the collateral. This
structure relies heavily on borrower repayment continuity rather than asset
recovery value.
In summary, the issuer has effectively traded \textbf{Collateral
Amortization Speed} for \textbf{Payment Affordability}. This structural choice has replaced
borrower-based risk with asset-based risk, a shift that is not reflected in the
"P-Series" Prime designation.
With the collateral effectively diluted, the portfolio faces inevitable
performance pressure. The following section demonstrates how the issuer actively
mitigates this deterioration to avoid contractual consequences.


\subsection{Strategic Management of Contractual Triggers}

\begin{figure}[H]
    \centering
    \includegraphics[width=0.85\textwidth]{plots/Carvana_Auto_Receivables_Trust_2021_P2_delinquency.png}
    \caption{Delinquency vs. Extensions: 60+ Day Delinquency Breach Analysis}
\end{figure}

Significant evidence of potential risk management techniques is found in the correlation 
between delinquency triggers and loan extensions.

\textbf{The Deferral Mechanism:} The issuer utilizes loan extensions 
to maintain the 60+ Day Delinquency rate below the 4.5\% threshold. 

\textbf{Forensic Analysis:} As of late 2025, the reported delinquency rate was 
\textbf{4.3\%}, yet the \textbf{Adjusted Delinquency Rate (inclusive of 
deferrals)} stood at \textbf{5.2\%}. This constitutes a \textbf{Technical Divergence} 
of the contractual Delinquency Trigger.

\textbf{The "Clustering" Evidence:} High-resolution analysis of the 2021-P2 
trust reveals a high-frequency clustering of extensions (grey bars) occurring 
immediately prior to the delinquency line (green) approaching the 4.5\% red 
trigger line. 

This pattern indicates that extensions are being deployed 
as a tactical response to avoid "Early Amortization" events. By masking the true 
loss severity, the issuer is delaying the mandatory "protective diversions" 
of cash flow required by the Trust Indenture, thereby prioritizing short-term 
performance metrics.
The behaviors documented in the 2021-P2 trust are not isolated anomalies.
Cross-trust analysis of the P-Series\footnote{Additional graphs included in the Appendix} reveals a consistent operational strategy:
when organic delinquency levels approach contractual triggers, extension volume
increases proportionately to maintain technical compliance while managing
economic performance.

\section{Conclusion and SEC Action Requested}
The Carvana P-Series exhibits a structural risk profile that has statistically diverged from its "Prime" designation. The evidence suggests that the issuer is navigating a \textbf{Collateral Performance Challenge} by utilizing mathematical levers—specifically term extensions and unverified income data—to maintain the appearance of performance while the underlying asset security erodes.

The "P-Series" branding currently serves as a mechanism for a portfolio characterized by deep negative equity and a decoupling of credit scores from risk-based pricing. If the divergence between reported delinquency and "Adjusted Delinquency" (inclusive of deferrals) continues to be managed via tactical extensions, the contractual protections intended for ABS investors will be rendered ineffective.

\textbf{To protect market integrity, we respectfully request:}
\begin{enumerate}
    \item \textbf{Targeted Audit of Stated Income:} A formal investigation into the validity of "Code 3" income entries, cross-referencing a statistically significant sample of the 2024 and 2025 vintages against IRS tax transcripts.
    \item \textbf{Review of Extension Policies:} An inquiry into whether the "clustering" of loan extensions prior to delinquency trigger breaches constitutes a violation of the Trust Indenture or a deceptive practice under the Securities Act.
    \item \textbf{Mandatory Disclosure of Adjusted LTV:} A requirement for the issuer to disclose "Current LTV" based on real-time vehicle depreciation curves rather than relying on original purchase prices in a declining used-car market.
    \item \textbf{Reclassification Assessment:} A review of the "Prime" designation for the P-Series to determine if the $R^2$ decoupling and term-extension profiles require more stringent capital reserve requirements.
\end{enumerate}

\section{Appendix}
\subsection{Data Sources}
\begin{itemize}
\item All data used in this report was sourced from 10-D and ABS-EE filings on
EDGAR. All statistics, figures, and tables were generated by analyzing data
submitted by Carvana to the SEC.

\item An LLM was used to parse delinquency and extension data from over 400 10-D
filings. The raw data created by the LLM was manually audited for accuracy and 
modified in a few places where the LLM misread the tables in the filings. Both 
the raw data and modified data files are included with this submission. 

\item All scripts used to pull data from EDGAR, to process the data, and to generate
the figures and tables in this report are included in the submission. Additionally, 
all of the code and scripts used to generate this report can be found on GitHub at:
\url{https://github.com/Abelian-Analysis/Carvana}.
\end{itemize}

\subsection{Data Hygiene}
The quantitative findings in this report are derived from a granular analysis of
Reg AB II loan-level data. To ensure the statistical relevance of risk metrics,
the following data hygiene protocol was implemented in R:

\begin{itemize}
    \item \textbf{Active Receivables Isolation:} The dataset was filtered to
    exclude liquidated and paid-off loans. Specifically, any record where the
    \texttt{reportingPeriodBeginningLoanBalanceAmount} was less than or equal to
    \$0.01 was removed.
    \item \textbf{Rationale for Filtering:} This exclusion is necessary to
    prevent "Survival Bias" from diluting point-in-time risk assessments.
    Including zero-balance loans would artificially lower the \textbf{Weighted
    Average LTV} and \textbf{Delinquency Percentages}, as these loans no longer
    contribute to the pool's exposure to loss.
    \item \textbf{Statistical Integrity:} By restricting the scope to active
    receivables, the resulting $R^2$ values and PTI trends reflect the
    \textbf{live economic reality} of the portfolio's current interest-bearing
    assets.
\end{itemize}

\subsection{A Pattern Of Trigger Management Across Vintages}

The strategic clustering of loan extensions immediately prior to delinquency
trigger breaches, as documented in the 2021-P2 trust analysis, is not an
isolated incident. Rather, it represents a systematic operational pattern that
repeats consistently across the entire P-Series portfolio spanning multiple
vintage years. The following graphs illustrate how this trigger management
mechanism—where extension volume increases proportionately as delinquency rates
approach the 4.5\% contractual threshold—appears across 2021, 2022, and
subsequent vintages. This cross-vintage consistency suggests an
institutionalized approach to managing contractual triggers, where the issuer
deploys deferrals as a tactical tool to maintain technical compliance while
masking the underlying economic deterioration of the portfolio. By
systematically preventing reported delinquency from exceeding contractual
thresholds, the issuer effectively circumvents the mandatory investor
protections encoded in the Trust Indentures, perpetuating the misalignment
between disclosed risk and economic reality discussed in the Strategic
Management section.

\begin{figure}[H]
    \centering
    \includegraphics[width=0.85\textwidth]{plots/Carvana_Auto_Receivables_Trust_2021_P1_delinquency.png}
\end{figure}
\begin{figure}[H]
    \centering
    \includegraphics[width=0.85\textwidth]{plots/Carvana_Auto_Receivables_Trust_2021_P2_delinquency.png}
\end{figure}
\begin{figure}[H]
    \centering
    \includegraphics[width=0.85\textwidth]{plots/Carvana_Auto_Receivables_Trust_2021_P3_delinquency.png}
\end{figure}
\begin{figure}[H]
    \centering
    \includegraphics[width=0.85\textwidth]{plots/Carvana_Auto_Receivables_Trust_2021_P4_delinquency.png}
\end{figure}
\begin{figure}[H]
    \centering
    \includegraphics[width=0.85\textwidth]{plots/Carvana_Auto_Receivables_Trust_2022_P1_delinquency.png}
\end{figure}
\begin{figure}[H]
    \centering
    \includegraphics[width=0.85\textwidth]{plots/Carvana_Auto_Receivables_Trust_2022_P2_delinquency.png}
\end{figure}
\begin{figure}[H]
    \centering
    \includegraphics[width=0.85\textwidth]{plots/Carvana_Auto_Receivables_Trust_2022_P3_delinquency.png}
\end{figure}



\end{document}